%was node [circle,draw=black,inner sep=0pt,minimum size={width("AAA")}] {\makeAlph{\x}\makeAlph{\y}};}}
% was [square,draw=black]
\tikzset{square/.style={rectangle,inner sep=0pt,minimum size=1cm}}

%\newcommand{\makeAlph}[1]{\symbol{\numexpr64+#1}}
%Only AlphAlph works, idk why
\newcommand{\makeAlph}[1]{\AlphAlph{#1}}
%\newcommand{\undoAlph}[1]{\numexpr`#1-`A\relax}
\newcommand{\undoAlph}[1]{\numexpr`#1-`A+1}

\providecommand{\flatten}[1]{
    \scalebox{0.8}{
    \begin{tikzpicture}
    %\draw[shift={(-0.5,0.5)}] (1,-1) grid (#1+1,-#1-1);
    %\draw[shift={(-0.5,0.5)}] (2*#1,-#1/2-0.5) grid +(#1*#1,-1);
        \foreach \x in {1,...,#1} {
            \foreach \y in {1,...,#1} {
                \pgfmathsetmacro\result{int((\y-1)*100/(#1-1))}
                \pgfmathsetmacro\counter{int((\y-1)*#1+\x)}
                \pgfmathsetmacro\updown{mod(\counter,2)}
            \node[square,draw=black,fill=gray!\result] at (\x,-\y) {$A_{\y\x}$};
                \node[square,draw=black,fill=gray!\result] at (#1+\y*#1+\x,-#1/2-0.5) {$A_{\y\x}$};
                \node at (#1+\y*#1+\x,-#1/2+0.5) {\counter};
        }}
        \draw[->,very thick] (#1+0.5+0.25,-#1/2-0.5) -- (2*#1+0.5-0.25,-#1/2-0.5);

    \end{tikzpicture}
}}

\providecommand{\toroid}[1]{
    \draw[shift={(-0.5,0.5)}] (1,-1) grid (#1+1,-#1-1);
        \foreach \x in {1,...,#1} {
            \foreach \y in {1,...,#1} {
                \node at (\x,-\y) {\makeAlph{\y}\makeAlph{\x}};}}
}
\providecommand{\superToroid}[4]{
    \IfBooleanTF#1{\providecommand{\shading}{white}}{\providecommand{\shading}{gray}}
    \draw[shift={(-0.5,0.5)}] (1,-1) grid (#2+1,-#2-1);
        \foreach \x in {1,...,#2} {
            \foreach \y in {1,...,#2} {
        \pgfmathtruncatemacro{\xpos}{(\y-1)*#2+(\x-1)}
        \pgfmathtruncatemacro{\ypos}{((#4-1)*#2+(#3-1))}
        \ifnum\xpos<\ypos
        \providecommand{\param}{white}
        \else
        \providecommand{\param}{\shading}
        \fi
\node[square,draw=black,fill=\param] at (\x,-\y)[align=center] {\makeAlph{#4}\makeAlph{#3}\\ \makeAlph{\y}\makeAlph{\x}};}}
}
\providecommand{\toroidWarp}[1]{
    \scalebox{0.8}{
    \begin{tikzpicture}
        \toroid{#1}
        \foreach \x in {1,...,#1} {
            \node at (\x,0) {\makeAlph{#1}\makeAlph{\x}};
            \node at (\x,-#1-1) {\makeAlph{#1}\makeAlph{\x}};
        }
        \foreach \y in {1,...,#1} {
            \node at (0,-\y) {\makeAlph{\y}\makeAlph{#1}};
            \node at (#1+1,-\y) {\makeAlph{\y}\makeAlph{1}};
        }
    \end{tikzpicture}
}}

\providecommand{\toroidAxes}[4]{
    \superToroid{#1}{#2}{#3}{#4}
        \foreach \x in {1,...,#2} {
            \node at (\x,0) {\makeAlph{\x}};
        }
        \foreach \y in {1,...,#2} {
            \node at (0,-\y) {\makeAlph{\y}};
        }
}

\NewDocumentCommand{\nestedToroid}{sm}{
    \scalebox{0.8}{
    \begin{tikzpicture}[scale=#2+1]
        \begin{scope}[scale=1]
            \draw[shift={(-0.5,0.5)}] (1,-1) grid (#2+1,-#2-1);
            \foreach \x in {1,...,#2} {
                \node at (\x,-0.25) {\makeAlph{\x}};
            }
            \foreach \y in {1,...,#2} {
                \node at (0.25,-\y) {\makeAlph{\y}};
            }
            \foreach \xa in {1,...,#2} {
                \foreach \ya in {1,...,#2} {
                    \begin{scope}[shift={({\xa-0.5+1/(#2+1)^2},{-\ya+0.5-1/(#2+1)^2})},scale={1/(#2+1)}]
                        \toroidAxes{#1}{#2}{\xa}{\ya}
                    \end{scope}
                }
            }
        \end{scope}
    \end{tikzpicture}
    }
}
%a
\NewDocumentCommand{\twodcomparison}{sm}{
    \IfBooleanTF#1{\providecommand{\shading}{white}}{\providecommand{\shading}{gray}}
    \scalebox{0.8}{
\begin{tikzpicture}
        %\draw[shift={(-0.5,0.5)}] (0,0) grid ({(#2^2)},{-(#2^2)});
        \foreach \x in {1,...,#2} {
            \foreach \xa in {1,...,#2} {
                \node at ({(\x-1)*#2+(\xa-1)},1) {\makeAlph{\x}\makeAlph{\xa}};
        }}
        \foreach \y in {1,...,#2} {
            \foreach \ya in {1,...,#2} {
                \node at (-1,{-((\y-1)*#2+(\ya-1))}) {\makeAlph{\y}\makeAlph{\ya}};
        }}
        \foreach \xa in {1,...,#2} {
            \foreach \ya in {1,...,#2} {
            \foreach \xb in {1,...,#2} {
                \foreach \yb in {1,...,#2} {
                \pgfmathtruncatemacro{\xpos}{(\xa-1)*#2+(\ya-1)}
                \pgfmathtruncatemacro{\ypos}{((\xb-1)*#2+(\yb-1))}
                %\pgfmathtruncatemacro\xpos{}
                %\pgfmathtruncatemacro\ypos{}
        \ifnum\xpos<\ypos
        \providecommand{\param}{white}
        \else
        \providecommand{\param}{\shading}
        \fi
\node[square,draw=black,fill=\param] at (\xpos,-\ypos)[align=center] {\makeAlph{\xb}\makeAlph{\yb}\\ \makeAlph{\xa}\makeAlph{\ya}};}}
            }
        }
\end{tikzpicture}
}}

\newcommand{\drawGrid}[2]{
    %1 is the length of the raw toroid
    %2 is the list
    \begin{tikzpicture}
    \foreach \token [count=\i from 0] in #2 {
        \pgfmathtruncatemacro{\x}{mod(\i,#1)}
        \pgfmathtruncatemacro{\y}{\i/#1}
    \node[square,draw=black] at (\x,-\y) {\token};}
    \end{tikzpicture}
}
\ExplSyntaxOn
\NewDocumentCommand{\superposition}{mm}{
    \scalebox{0.8}{
\begin{tikzpicture}
        %\draw[shift={(-0.5,0.5)}] (0,0) grid ({(#2^2)},{-(#2^2)});
        \foreach \x in {1,...,#1} {
            \foreach \xa in {1,...,#1} {
                \node at ({(\x-1)*#1+(\xa-1)},1) {\makeAlph{\x}\makeAlph{\xa}};
        }}
        \foreach \y in {1,...,#1} {
            \foreach \ya in {1,...,#1} {
                \node at (-1,{-((\y-1)*#1+(\ya-1))}) {\makeAlph{\y}\makeAlph{\ya}};
        }}
        \foreach \xa in {1,...,#1} {
            \foreach \ya in {1,...,#1} {
            \foreach \xb in {1,...,#1} {
                \foreach \yb in {1,...,#1} {
                \pgfmathtruncatemacro{\xpos}{(\xa-1)*#1+(\ya-1)}
                \pgfmathtruncatemacro{\ypos}{((\xb-1)*#1+(\yb-1))}
            \str_if_eq:eeTF {\makeAlph{\xb}\makeAlph{\yb}} {\clist_item:nn {#2}{(\xa-1)*#1+\ya}}
            {\providecommand{\param}{gray}}
            {\providecommand{\param}{white}}
\node[square,draw=black,fill=\param] at (\xpos,-\ypos)[align=center] {\makeAlph{\xb}\makeAlph{\yb}\\ \makeAlph{\xa}\makeAlph{\ya}};}}
            }
        }
\end{tikzpicture}
}}
\ExplSyntaxOff
