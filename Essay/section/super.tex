
\section{Superposition}%
\label{sec:superposition}
Since the entries into the toroidal grid are discrete (eg. $AA$ resolves to discrete coordinates within the grid), it is not possible to optimize the loss function. Therefore, relaxing the constraints to enable superposition -- here defined as having a token being in multiple positions at once each with its own ``probabilities'' -- is essential. In this essay, ``probability'' will not refer to the likeliness of a random event, but rather, the confidence of a token in its posiiton.

A simple method of allowing superposition, is by allowing any position to have any token value, which again, can be visualized as a 4-dimensional matrix, and 2-dimensional matrix, as seen in \autoref{fig:superposition}. But this time, the elements of the grid do not represent the distances between tokens. But rather, they represent the probability of a token being placed in a certain position. Further constraints to limit the total probability to 1 will be discussed in a later section. Note that now the $y-$axis represents various positions (the same token in various places), and the $x-$axis represents the possible token values (all of which lie on the same toroid grid cell).
\begin{figure}[htpb]
    \centering
    \begin{subfigure}[t]{0.5\textwidth}
    \begin{center}
    \nestedToroid*{2}
    \end{center}
    \caption{$A_{ijkl}$, a 4-dimensional superposition}
    \end{subfigure}%
    ~
    \begin{subfigure}[t]{0.5\textwidth}
    \begin{center}
    \twodcomparison*{2}
    \end{center}
    \caption{$A_{ij,kl}$ a 2-dimensional superposition}
    \end{subfigure}

    \caption{Tensors $S$ representing superpositions of a $2\times 2$ toroidal grid, where every element $S_{ijkl}$ represents the probability of the token $KL$ being in the of position $IJ$ in the original grid}%
    \label{fig:superposition}
\end{figure}

An example is shown in \autoref{fig:superpositionExample}

\begin{figure}[htpb]
    \centering
    \begin{subfigure}[t]{0.5\textwidth}
    \begin{center}
\drawGrid{2}{{BB,AB,BA,AA}}
    \end{center}
    \caption{$A_{ijkl}$, an example toroidal grid}
    \end{subfigure}%
    ~
    \begin{subfigure}[t]{0.5\textwidth}
    \begin{center}
\superposition{2}{BB,AB,BA,AA}
    \end{center}
    \caption{$A_{ij,kl}$ superposition of the grid on the left}
    \end{subfigure}

    \caption{Shaded cells represent cells with probability 1, the rest have probability 0}
    \label{fig:superpositionExample}
\end{figure}

\subsubsection{Generalization of the Loss Function}%
\label{ssub:generalization_of_the_loss_function}

Defining the loss function for this formulation requires us to compare every value, in every position, to every value in every position. And we must do so, while taking both of their confidence's into account (ie. the distance metric should be scaled by their confidence). Therefore, the distance should be scaled with
\begin{equation}
    S_{ab}S_{cd}
\end{equation}

The distance between these two probabilities is defined as $C_{a,c}$, because only the indices $a$ and $c$ correspond to positions on the grid (thus being important in calculating distance), whereas $b$ and $d$ correspond to only the token itself.

Alike with the situation in \autoref{ssub:token_comparisons}, we must prevent duplicate comparisons between values (not position), for example: each of %
[$\begin{smallmatrix}AA\\AA\end{smallmatrix}$,%
$\begin{smallmatrix}AB\\AA\end{smallmatrix}$,%
$\begin{smallmatrix}BA\\AA\end{smallmatrix}$,%
$\begin{smallmatrix}BB\\AA\end{smallmatrix}$] should be compared to
[$\begin{smallmatrix}AA\\AB\end{smallmatrix}$,%
$\begin{smallmatrix}AB\\AB\end{smallmatrix}$,%
$\begin{smallmatrix}BA\\AB\end{smallmatrix}$,%
$\begin{smallmatrix}BB\\AB\end{smallmatrix}$], but not vice versa, because the values (not the positions) will have been compared already. Again, to do so, we must construct an upper triangular matrix $U$, of the shape $N^2\times N^2$.

Therefore, the loss function can be written as
\begin{equation}
    L(X)=\sum_{a}^{N^2} \sum_{b}^{N^2} \sum_{c}^{N^2} \sum_{d}^{N^2} S_{a,b} S_{c,d} C_{a,c}C_{b,d}U_{a,c}
\end{equation}

\begin{itemize}
    \item [$\sum_{a}^{N^2}$] Represents an iteration over Source Positions
    \item [$\sum_{b}^{N^2}$] Represents an iteration over Source Values
    \item [$\sum_{c}^{N^2}$] Represents an iteration over Target Positions
    \item [$\sum_{d}^{N^2}$] Represents an iteration over Target Values
    \item [$S_{a,b}$] Represents the Probability of the Source Position and Value
    \item [$S_{c,d}$] Represents the Probability of the Target Position and Value
    \item [$C_{a,c}$] Represents the Distance between Source and Target Position
    \item [$C_{b,d}$] Represents the Distance between the Source and Target Values in the original grid
    \item [$U_{a,c}$] Upper triangular matrix to remove redundant value comparisons
\end{itemize}
